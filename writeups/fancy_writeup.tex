%%%%%%%%%%%%%%%%%%%%%%%%%%%%%%%%%%%%%%%%%
% Large Colored Title Article
% LaTeX Template
% Version 1.1 (25/11/12)
%
% This template has been downloaded from:
% http://www.LaTeXTemplates.com
%
% Original author:
% Frits Wenneker (http://www.howtotex.com)
%
% License:
% CC BY-NC-SA 3.0 (http://creativecommons.org/licenses/by-nc-sa/3.0/)
%
%%%%%%%%%%%%%%%%%%%%%%%%%%%%%%%%%%%%%%%%%

%----------------------------------------------------------------------------------------
%PACKAGES AND OTHER DOCUMENT CONFIGURATIONS
%----------------------------------------------------------------------------------------

\documentclass[DIV=calc, paper=a4, fontsize=12pt, twocolumn]{scrartcl} % A4 paper and 11pt font size

\setlength{\columnsep}{25pt}

\usepackage{hyperref}
\usepackage[english]{babel} % English language/hyphenation
\usepackage[protrusion=true,expansion=true]{microtype} % Better typography
\usepackage{amsmath,amsfonts,amsthm} % Math packages
\usepackage[svgnames]{xcolor} % Enabling colors by their 'svgnames'
\usepackage[hang, small,labelfont=bf,up,textfont=it,up]{caption} % Custom captions under/above floats in tables or figures
\usepackage{booktabs} % Horizontal rules in tables
\usepackage{fix-cm} % Custom font sizes - used for the initial letter in the document

\usepackage{sectsty} % Enables custom section titles
\allsectionsfont{\usefont{OT1}{phv}{b}{n}} % Change the font of all section commands

\usepackage{setspace}

\usepackage{fancyhdr} % Needed to define custom headers/footers
\pagestyle{fancy} % Enables the custom headers/footers
\usepackage{lastpage} % Used to determine the number of pages in the document (for ``Page X of Total'')

% Headers - all currently empty
\lhead{}
\chead{}
\rhead{}

% Footers
\lfoot{}
\cfoot{}
\rfoot{\footnotesize Page \thepage\ of \pageref{LastPage}} % ``Page 1 of 2''

\renewcommand{\headrulewidth}{0.0pt} % No header rule
\renewcommand{\footrulewidth}{0.4pt} % Thin footer rule

\usepackage{lettrine} % Package to accentuate the first letter of the text
\newcommand{\initial}[1]{ % Defines the command and style for the first letter
\lettrine[lines=3,lhang=0.3,nindent=0em]{
\color{DarkGoldenrod}
{\textsf{#1}}}{}}

%----------------------------------------------------------------------------------------
%TITLE SECTION
%----------------------------------------------------------------------------------------

\usepackage{titling} % Allows custom title configuration

\newcommand{\HorRule}{\color{DarkGoldenrod} \rule{\linewidth}{1pt}} % Defines the gold horizontal rule around the title

\pretitle{\vspace{-30pt} \begin{flushleft} \HorRule \fontsize{30}{30} \usefont{OT1}{phv}{b}{n} \color{DarkRed} \selectfont} % Horizontal rule before the title

\title{An exploration into the determinants of living standards} % Your article title

\posttitle{\par\end{flushleft}\vskip 0.5em} % Whitespace under the title

\preauthor{\begin{flushleft}\large \lineskip 0.5em \usefont{OT1}{phv}{b}{sl} \color{DarkRed}} % Author font configuration

\author{Kevin Chen '15, Varun Sharma '16, Jean-Luc Etienne '15, } % Your name

\postauthor{\footnotesize \usefont{OT1}{phv}{m}{sl} \color{Black} % Configuration for the institution name
Williams College % Your institution

\par\end{flushleft}\HorRule} % Horizontal rule after the title

\date{} % Add a date here if you would like one to appear underneath the title block

%----------------------------------------------------------------------------------------

\begin{document}

\maketitle % Print the title
\doublespacing
\thispagestyle{fancy} % Enabling the custom headers/footers for the first page 




% % % % % % % % % % %
% % % % % % % % % % %
% % % % % % % % % % %
\initial{T}\textbf{he determinants of a nation's gross domestic product (GDP) have been explored to a great extent.
Indeed, given that GDP is a function of a nation's
consumption, investment, government spending, exports, and imports,
the standard Keynesian model ``predicts'' GDP nearly perfectly, and quite uninterestingly.
We present four core models that try to explain GDP per capita without
relying on the literal components of Keynesian aggregate demand.
Specifically, they proxy for measures of climate, technological advancement,
urbanization and urban welfare, and social development.
We find that\dots}



% % % % % % % % % % %
% % % % % % % % % % %
% % % % % % % % % % %
\section{Introduction}
What makes one country wealthier, or more well off, than another?
In Keynesian theory, GDP can be reduced to its literal components:
$$\mathrm{GDP} = C + I + G + (X - M)$$
where 
$C$ is household expenditure, 
$I$ is investment, 
$G$ is government spending,
$X$ is exports, and
$M$ is imports.

However, regressing GDP on these components would be pointless as it would confirm nothing new,
and uninteresting as it would not be an appropriate proxy for a country's standard of living.
Instead, we use a more telling response and more interesting indicators.

\subsection{Why ``per-capita'' GDP?}
GDP is not a suitable proxy for standards of living.
For proof of this, look no further than China.
There, GDP has skyrocketed while GDP per capita remains one of the world's lowest. 
Although GDP per capita may not be a perfect metric for living standards -- 
since not all citizens benefit from a country's increased production -- 
it has been argued that GDP per capita tends to move with living standards.
In addition, most countries in the world frequently publish GDP per capita data.
Wide access to data, as well as the fact that per-capita GDP is measured 
with a relatively consistent definition, makes this a suitable proxy for a
country's standards of living.

\subsection{Research Question}
Three out of four of our models are inspired from three simple icons:
climate change, cities, and computers.
These models beg the following question:
Are people in greener, more urbanized, and more technologically advanced countries better off?

It is difficult to hypothesize what effect each of these indicator sets will have on a per-capita response.
It makes sense to think that a country with higher pollution is more industrial and thus has a higher output, 
but how would that affect per-capita output? This difficulty is present for our urbanization and technology models as well.
A more urbanized nation may experience the agglomeration economies inherent in cities, and a more
tech-savvy country may reap productivity benefits from knowledge spillovers, but the effect on per-capita GDP is less clear.

Our fourth and final model bundles together various indices for social development,
such as literacy rates, life expectancy, refugee population, and child labor.

For all four models, we hypothesize that each indicator is positively correlated with the reponse.




% % % % % % % % % % %
% % % % % % % % % % %
% % % % % % % % % % %
\section{Data}
We collect data from the \href{http://data.worldbank.org/indicator}{World Development Indicators}, 
which draws from several internationally recognized sources and boasts 54 years' worth of data for 214 countries across 1334 indicators.
We use data for the year 2008.

For modularity, we built a script that created a data frame for each year of data downloaded, and then recast each frame from wide to long format.

details of how data was collected, W's and missing data




% % % % % % % % % % %
% % % % % % % % % % %
% % % % % % % % % % %
\section{Analysis}
exploratory analysis, checking model conditions and how we changed our model based on that



% % % % % % % % % % %
% % % % % % % % % % %
% % % % % % % % % % %
\section{Results}
present fitted model and inferences about research question



% % % % % % % % % % %
% % % % % % % % % % %
% % % % % % % % % % %
\section{Conclusion}
summarize results, describe limitations and how they might be addressed


\end{document}
