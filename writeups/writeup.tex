\documentclass[12pt, twocolumn]{article}

\usepackage{usenix}
\usepackage{amssymb}
\usepackage{amsmath}
\usepackage{graphicx}
\usepackage{setspace}
\usepackage{hyperref}
\usepackage{endnotes}

\title{
  \sc
  What Makes a Country More Well Off? \\
  A Regression Analysis of GDP per capita
}

\author{
    {\rm Kevin Chen '15} \\ 
    \and
    {\rm Varun Sharma '16} \\ 
    \and
    {\rm Jean-Luc Etienne '15} \\ 
}

\hypersetup{
    bookmarks=true,         % show bookmarks bar?
    unicode=false,          % non-Latin characters in Acrobat’s bookmarks
    pdftoolbar=true,        % show Acrobat’s toolbar?
    pdfmenubar=true,        % show Acrobat’s menu?
    pdffitwindow=false,     % window fit to page when opened
    pdfstartview={FitH},    % fits the width of the page to the window
    pdftitle={My title},    % title
    pdfauthor={Author},     % author
    pdfsubject={Subject},   % subject of the document
    pdfcreator={Creator},   % creator of the document
    pdfproducer={Producer}, % producer of the document
    pdfkeywords={keyword1} {key2} {key3}, % list of keywords
    pdfnewwindow=true,      % links in new window
    colorlinks=true,       % false: boxed links; true: colored links
    linkcolor=red,          % color of internal links (change box color with linkbordercolor)
    citecolor=green,        % color of links to bibliography
    filecolor=magenta,      % color of file links
    urlcolor=cyan           % color of external links
}

\begin{document}

\date{}
\maketitle
\doublespacing


% % % % % % % % % % %
% % % % % % % % % % %
% % % % % % % % % % %
\subsection*{\sc Abstract}
The determinants of a nation's gross domestic product (GDP) have been explored to a great extent. Indeed, given that GDP is a function of a nation's consumption, investment, government spending, exports, and imports, the standard Keynesian model ``predicts'' GDP nearly perfectly, and quite uninterestingly. We present four core models that try to explain GDP per capita without relying on the literal components of Keynesian aggregate demand. Specifically, they proxy for measures of climate, technological advancement, urbanization and urban welfare, and social development.
We find that\dots %TODO



% % % % % % % % % % %
% % % % % % % % % % %
% % % % % % % % % % %
\section{Introduction}
What makes one country wealthier, or more well off, than another? In Keynesian theory, GDP can be reduced to its literal components:
$$\mathrm{GDP} = C + I + G + (X - M)$$
where $C$ is household expenditure, $I$ is investment,  $G$ is government spending, $X$ is exports, and $M$ is imports.

However, regressing GDP on these components would be pointless as it would confirm nothing new, and uninteresting as it would not be an appropriate proxy for a country's standard of living. Instead, we use a more telling response and more interesting indicators.

\subsection{Why ``per-capita'' GDP?}
GDP is not a suitable proxy for standards of living. For proof of this, look no further than China. There, GDP has skyrocketed while GDP per capita remains one of the world's lowest. Although GDP per capita may not be a perfect metric for living standards -- since not all citizens benefit from a country's increased production -- it has been argued that GDP per capita tends to move with living standards. In addition, most countries in the world frequently publish GDP per capita data. Wide access to data, as well as the fact that per-capita GDP is measured with a relatively consistent definition, makes this a suitable proxy for a country's standards of living.

\subsection{Research Question}
Our models are inspired from four simple buzzwords: \emph{climate change}, \emph{cities}, \emph{computers}, and \emph{``comfort''} (used somewhat euphemistically). These models beg the following question: Are people better off in greener, more urbanized, more technologically advanced, and more socially developed countries?

For our climate change model, we use metrics for energy consumption, pollution, and paved roads -- the idea being that firm production rises with energy consumption, that pollution spikes as a result, and that firms are more likely to produce if they face less congestion when transporting their goods.

For our urbanization model, we use proxies for urban welfare and urbanization level. Cities provide benefits of agglomeration for both firms and workers. In a city, a worker is more likely to match with a firm, to find employment, and to spread knowledge between firms. Similarly, when firms cluster together in a city, they experience greater knowledge spillovers, lower proximity from their buyers, and thus lower transportation costs. In addition, we hypothesize that wealthier urban populations will experience greater productivity and thus lead to greater national output.

Next, for our technology model, we use metrics for high-tech exports, and research and development (R\&D). We hypothesize that these metrics are directly correlated with innovation and productivity, and that they rise with GDP as a result.

Finally, for our social development model, we look at the size of a country's refugee population, the preponderance of child labor, literacy rates, and life expectancy. We hypothesize that literacy rates and life expectancy are positively correlated with GDP, as countries that are healthier and more educated tend to have healthier economies. We also hypothesize that refugee population and child labor are negatively correlated with national output, as they are indicative of poverty, poor schooling, and nearby civil strife.

Given these models, we face several challenges. First, while we have provided hypotheses for each indicator's effect on GDP, we have not hypothesized about their effects on per-capita GDP. That is, a country with higher pollution may be more industrial and thus have a higher output, but what does that say about its population size and consequently about its GDP per capita?

Second, we may face multicollinearity in our models. For instance, our social development model includes literacy rates and child labor. The former is a proxy for education, while the latter is often linked directly to education. This is problematic since multicollinearity increases standard error and decreases an indicator's $t$-statistic, thus making it appear insignificant when in reality it is not.




% % % % % % % % % % %
% % % % % % % % % % %
% % % % % % % % % % %
\section{Data}
We collect data from the \href{http://data.worldbank.org/indicator}{World Development Indicators}, 
which draws from several internationally recognized sources and boasts 54 years' worth of data for 214 countries across 1334 indicators.
We use data for the year 2008.\endnotemark[1]

We use heat maps to determine which indicators had few missing data points, including our response, as shown in Figure~\ref{gdp_per_capita_heat_map}.

\begin{figure}[h!]
\centering
%\includegraphics[scale=0.5]{images/gdp_per_capita_heat_map.jpg}
\caption{\label{gdp_per_capita_heat_map}Our response had near-universal coverage.}
\end{figure}



% % % % % % % % % % %
% % % % % % % % % % %
% % % % % % % % % % %
\section{Analysis}
As with any linear regression model, we provide exploratory analysis of our assumptions: linearity, independence, zero mean, constant variance, and normality.

By virtue of using ordinary least squares linear regression, the mean error is zero. In addition, we safely assume independence, as the collection of data points was done in such a way that one datum did not influence another.

To confirm linearity, we provide scatterplots of GDP per capita against each indicator.

To confirm constance variance, we provide scatterplots of residuals versus fitted values.

To confirm the normality of the error distribution, we provide a Q-Q plot for each of our models.

Furthermore, we fix each violated condition by transforming the indicator in question or by removing it from the model entirely.

\subsection{Climate Change Model Conditions}

\begin{figure*}[h!]
  \centering
  \includegraphics[width=\textwidth]{images/climate_model_conditions}
  \caption{\label{climate_model_conditions}Residuals vs fitted and Q-Q plot}
\end{figure*}

\subsection{Cities Model Conditions}
\subsection{Computers Model Conditions}
\subsection{Social Development Model Conditions}



% % % % % % % % % % %
% % % % % % % % % % %
% % % % % % % % % % %
\section{Results}
present fitted model and inferences about research question



% % % % % % % % % % %
% % % % % % % % % % %
% % % % % % % % % % %
\section{Conclusion}
summarize results, describe limitations and how they might be addressed


\endnotetext[1]{For modularity, we built a script that created a data frame for each year of data downloaded, and then recast each frame from wide to long format.}

\theendnotes

\end{document}
