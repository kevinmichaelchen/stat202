\documentclass[12pt, twocolumn]{article}

\usepackage{usenix}
\usepackage{amssymb}
\usepackage{amsmath}
\usepackage{graphicx}
\usepackage{setspace}
\usepackage{hyperref}

\title{
  \sc
  What Makes a Country More Well Off? \\
  A Regression Analysis of GDP per capita
}

\author{
    {\rm Kevin Chen '15} \\ 
    \and
    {\rm Varun Sharma '16} \\ 
    \and
    {\rm Jean-Luc Etienne '15} \\ 
}

\hypersetup{
    bookmarks=true,         % show bookmarks bar?
    unicode=false,          % non-Latin characters in Acrobat’s bookmarks
    pdftoolbar=true,        % show Acrobat’s toolbar?
    pdfmenubar=true,        % show Acrobat’s menu?
    pdffitwindow=false,     % window fit to page when opened
    pdfstartview={FitH},    % fits the width of the page to the window
    pdftitle={My title},    % title
    pdfauthor={Author},     % author
    pdfsubject={Subject},   % subject of the document
    pdfcreator={Creator},   % creator of the document
    pdfproducer={Producer}, % producer of the document
    pdfkeywords={keyword1} {key2} {key3}, % list of keywords
    pdfnewwindow=true,      % links in new window
    colorlinks=true,       % false: boxed links; true: colored links
    linkcolor=red,          % color of internal links (change box color with linkbordercolor)
    citecolor=green,        % color of links to bibliography
    filecolor=magenta,      % color of file links
    urlcolor=cyan           % color of external links
}

\begin{document}

\date{}
\maketitle
\doublespacing


% % % % % % % % % % %
% % % % % % % % % % %
% % % % % % % % % % %
\subsection*{\sc Abstract}
The determinants of a nation's gross domestic product (GDP) have been explored to a great extent.
Indeed, given that GDP is a function of a nation's 
consumption, investment, government spending, exports, and imports, 
the standard Keynesian model ``predicts'' GDP nearly perfectly, and quite uninterestingly.
We present four core models that try to explain GDP per capita without 
relying on the literal components of Keynesian aggregate demand.
Specifically, they proxy for measures of climate, technological advancement, 
urbanization and urban welfare, and social development.
We find that\dots %TODO




% % % % % % % % % % %
% % % % % % % % % % %
% % % % % % % % % % %
\section{\sc Introduction}
What makes one country wealthier, or more well off, than another?
In Keynesian theory, GDP can be reduced to its literal components:
$$\mathrm{GDP} = C + I + G + (X - M)$$
where 
$C$ is household expenditure, 
$I$ is investment, 
$G$ is government spending,
$X$ is exports, and
$M$ is imports.

However, regressing GDP on these components would be pointless as it would confirm nothing new,
and uninteresting as it would not be an appropriate proxy for a country's standard of living.
Instead, we use a more telling response and more interesting indicators.

\subsection{\sc Why ``per-capita'' GDP?}
GDP is not a suitable proxy for standards of living.
For proof of this, look no further than China.
There, GDP has skyrocketed while GDP per capita remains one of the world's lowest. 
Although GDP per capita may not be a perfect metric for living standards -- 
since not all citizens benefit from a country's increased production -- 
it has been argued that GDP per capita tends to move with living standards.
In addition, most countries in the world frequently publish GDP per capita data.
Wide access to data, as well as the fact that per-capita GDP is measured 
with a relatively consistent definition, makes this a suitable proxy for a
country's standards of living.

\subsection{\sc Research Question}
Three out of four of our models are inspired from three simple icons:
climate change, cities, and computers.
These models beg the following question:
Are people in greener, more urbanized, and more technologically advanced countries better off?

It is difficult to hypothesize what effect each of these indicator sets will have on a per-capita response.
It makes sense to think that a country with higher pollution is more industrial and thus has a higher output, 
but how would that affect per-capita output? This difficulty is present for our urbanization and technology models as well.
A more urbanized nation may experience the agglomeration economies inherent in cities, and a more
tech-savvy country may reap productivity benefits from knowledge spillovers, but the effect on per-capita GDP is less clear.

Our fourth and final model bundles together various indices for social development,
such as literacy rates, life expectancy, refugee population, and child labor.

For all four models, we hypothesize that each indicator is positively correlated with the reponse.




% % % % % % % % % % %
% % % % % % % % % % %
% % % % % % % % % % %
\section{\sc Data}
We collect data from the \href{http://data.worldbank.org/indicator}{World Development Indicators}, 
which has 54 years' worth of data for 214 countries across 1334 indicators.
We use data for the year 2008.\endnote{For modularity, we built a script that created a data frame for each year of data downloaded, and then recast each frame from wide to long format.}
details of how data was collected, W's and missing data




% % % % % % % % % % %
% % % % % % % % % % %
% % % % % % % % % % %
\section{\sc Analysis}
exploratory analysis, checking model conditions and how we changed our model based on that



% % % % % % % % % % %
% % % % % % % % % % %
% % % % % % % % % % %
\section{\sc Results}
present fitted model and inferences about research question



% % % % % % % % % % %
% % % % % % % % % % %
% % % % % % % % % % %
\section{\sc Conclusion}
summarize results, describe limitations and how they might be addressed


\theendnotes

\end{document}
